% Created 2013-12-13 Fri 14:51
\documentclass[]{article}
\usepackage[utf8]{inputenc}
\usepackage[T1]{fontenc}
\usepackage{fixltx2e}
\usepackage{graphicx}
\usepackage{longtable}
\usepackage{float}
\usepackage{wrapfig}
\usepackage{rotating}
\usepackage[normalem]{ulem}
\usepackage{amsmath}
\usepackage{textcomp}
\usepackage{marvosym}
\usepackage{wasysym}
\usepackage{amssymb}
\usepackage{hyperref}
\tolerance=1000
\author{Daniele Barchiesi}
\date{\textit{<2013-12-13 Fri>}}
\title{Classification via dictionary learning: incoherent sub-space modelling for high-dimensional data}
\hypersetup{
  pdfkeywords={},
  pdfsubject={},
  pdfcreator={Emacs 24.3.1 (Org mode 8.2.4)}}
\begin{document}

\maketitle
\begin{abstract}

\end{abstract}

\section{Introduction}
\label{sec-1}
\begin{enumerate}
\item $\square$ prepare a reading list
\item $\square$ read papers
\item $\square$ select relevant publications
\item $\square$ write introduction and summaries
\end{enumerate}
\subsection{Problem definition}
\label{sec-1-1}

\subsection{Previous work and applications}
\label{sec-1-2}

\subsection{Outline and main contributions}
\label{sec-1-3}

\section{Background}
\label{sec-2}
\subsection{Benchmark techniques}
\label{sec-2-1}
\subsubsection{Features transform for classification}
\label{sec-2-1-1}
Feature transforms have been used to model data in an attempt to enhance the tradeoff between generalisation and discrimination, as described in Section \ref{sec-1}.
Two of the main feature transform techniques include principal component analysis (PCA) and Fisher's linear discriminant analysis (LDA). This section provides a brief description of their rationale.

Let $\{x_m \in R^N\}$ be a set of vectors containing features extracted from $M$ training signals.

\subsubsection{Manifold learning}
\label{sec-2-1-2}

\subsection{Dictionary learning}
\label{sec-2-2}
\subsubsection{Dictionary learning for sparse approximation}
\label{sec-2-2-1}
\subsubsection{Incoherent dictionary learning}
\label{sec-2-2-2}
\subsubsection{Dictionary learning for classification}
\label{sec-2-2-3}

\section{Learning incoherent subspaces using iterative projections and rotations}
\label{sec-3}
\section{Numerical Experiments}
\label{sec-4}
\section{Discussion}
\label{sec-5}
\section{Conclusions}
\label{sec-6}
% Emacs 24.3.1 (Org mode 8.2.4)
\end{document}
